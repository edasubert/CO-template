\documentclass{article}

\usepackage{lipsum}
\usepackage[utf8]{inputenc}
\usepackage[IL2]{fontenc}
\usepackage[czech]{babel}

\title{Jak sázet \textbf{Corpus Omne} s balíčkem \texttt{CorpusOmne}}
\author{Eduard Šubert}

\begin{document}

\maketitle

\begin{abstract}
Dokumentace k balíčku \texttt{CorpusOmne}, který vznikl na začátku roku 2014 na FJFI ČCUT v Praze pro sázení fakultního časopisu Corpus Omne. Postupně jsou vysvětleny všechny příkazy a prostředí přidané balíčkem, včetně velkého příkladu na konci. Případné dotazy a připomínky směrujte na \texttt{suberedu@fjfi.cvut.cz}.
\end{abstract}

\section{Titulní strana}%%%%%%%%%%%%%%%%%%%%%%%%%%%%%%%%%%%
Prostředí \texttt{COtitlepage} vysází titulní stranu. Parametr \texttt{Pozadí} je obrázek na pozadí (poměr stran je zachován), volitelný parametr \texttt{Offset} posunuje obrázek vertikálně.

Příkaz \texttt{COtitle} vysází hlavní titulek. Parametr \texttt{Datum} slouží k označení vydání. Měl by být používán výhradně v prostředí \texttt{COtitlepage}.

Příkaz \texttt{COtitleBottom} vysází spodní citaci. Měl by být používán výhradně v prostředí \texttt{COtitlepage}. 
Mezi \texttt{COtitle} a \texttt{COtitleBottom} může být libovolný obsah. 
\begin{verbatim}
\begin{COtitlepage}[Offset]{Pozadí}
\COtitle{Datum}
\vfill
\COtitleBottom{Citace}{Autor}
\end{COtitlepage}
\end{verbatim}
%
\section{Článek}%%%%%%%%%%%%%%%%%%%%%%%%%%%%%%%%%%%
Pro vysázení celého článku musíme použít několik příkazů:\\[0.3cm]
Příkaz \texttt{COarticle} vysází formátovaný nadpis \texttt{Nadpis} článku. (Přidá položku do Obsahu.)
\begin{verbatim}
\COarticle{Nadpis}
\end{verbatim}
Příkaz \texttt{COsilentarticle} vysází formátovaný nadpis \texttt{Nadpis} článku. (NEpřidá položku do Obsahu.)
\begin{verbatim}
\COsilentArticle{Nadpis}
\end{verbatim}
Prostředí \texttt{COcolumn} slouží pro sázení textu do sloupců. Nepovinný parametr \texttt{PočetSloupců} nastavuje počet sloupců. Výchozí hodnota je $3$.
\begin{verbatim}
\begin{COcolumn}[PočetSloupců]
\end{COcolumn}
\end{verbatim}
Příkaz \texttt{COquote} vysází zvýrazněnou citaci \texttt{Citace}. Měl by být používán výhradně v prostředí \texttt{COcolumn}.
\begin{verbatim}
\COquote{Citace}
\end{verbatim}
Příkaz \texttt{COpicture} vysází obrázek \texttt{Obrázek} s titulkem \texttt{Titulek}. Měl by být používán výhradně v prostředí \texttt{COcolumn}.
\begin{verbatim}
\COpicture{Obrázek}{Titulek}
\end{verbatim}
Příkaz \texttt{COauthor} vysází zkratku \texttt{Zkratka}. (Přidá položku do tiráže.) Měl by být používán výhradně před koncem prostředí \texttt{COcolumn}.
\begin{verbatim}
\COauthor{Jméno}{Zkratka}
\end{verbatim}
%
\section{Editorial}%%%%%%%%%%%%%%%%%%%%%%%%%%%%%%%%%%%
Editorial se sází stejně jako článek s jednou výjimkou:\\[0.3cm]
Příkaz \texttt{COeditorial} vysází nadpis. (NEpřidá položku do Obsahu.)
\begin{verbatim}
\COeditorial
\end{verbatim}
%
\section{Obsah}%%%%%%%%%%%%%%%%%%%%%%%%%%%%%%%%%%%
Obsah je generován automaticky, každý článek je na úrovni \texttt{subsection}. Je tedy Nutné přidat dělení \texttt{section}:\\[0.3cm]
Příkaz \texttt{COsection} NEvysází nic. (přidá položku do Obsahu.)
\begin{verbatim}
\COsection{Nadpis}
\end{verbatim}
Příkaz \texttt{COtableofcontents} vysází Obsah.
\begin{verbatim}
\COtableofcontents
\end{verbatim}
%
\section{Tiráž}%%%%%%%%%%%%%%%%%%%%%%%%%%%%%%%%%%%
Tiráž se chová podobně jako Obsah, tedy obsahuje úrovně \texttt{section} (kategorie) a \texttt{subsection} (autoři).
Příkaz \texttt{COLOAcategory} NEvysází nic. (přidá novou kategorii do Tiráže.)
\begin{verbatim}
\COLOAcategory{Kategorie}
\end{verbatim}
Příkaz \texttt{COsilentAuthor} NEvysází nic. Přidá do Tiráže autora bez zkratky.
\begin{verbatim}
\COsilentAuthor{Jméno}
\end{verbatim}
Příkaz \texttt{COmasthead} vysází Tiráž.
\begin{verbatim}
\COmasthead
\end{verbatim}
%
\section{Ostatní}%%%%%%%%%%%%%%%%%%%%%%%%%%%%%%%%%%%
\subsection{Obrázky}
Příkaz \texttt{COpagePicture} vysází obrázek \texttt{Obrázek} na celou stranu. Obrázek je roztažen až do krajů, musí mít poměr stran 210:297.
\begin{verbatim}
\COpagePicture{Obrázek}
\end{verbatim}
Příkaz \texttt{COfloatPicture} vysází obtékaný obrázek \texttt{Obrázek} s titulkem \texttt{Titulek}, zarovnáním \texttt{Zarovnání} $ \in \left[r,l\right] $ a šířkou \texttt{Šířka} (doporučuje se požívat tvar 0.4\\textwidth). Parametr \texttt{PočetObtékanýchŘádků} je výška obrázku v počtu řádků (více parametr \texttt{lineheight} prostředí \texttt{wrapfigure}).
\begin{verbatim}
\COfloatPicture{Obrázek}{Titulek}{Zarovnání}{Šířka}{PočetObtékanýchŘákdů}
\end{verbatim}
%
\section{Příklad}
%
\begin{verbatim}
\documentclass{CorpusOmne}

\usepackage{lipsum} %pro sázení textu lipsum

\begin{document}
\begin{COtitlepage}[70]{titulka.jpg}
\COtitle{(2) 2013/2014}
\vfill
Nějaký text bla bla\hfill
\vfill
\hfill ještě trochu textu
\vfill
\COtitleBottom{... a tuto rovnici vyřešíme metodou uhodnutí.}
	{prof. Ing. Jiří Tolar, DrSc.}
\end{COtitlepage}

\COpagePicture{CO_vanoce.jpg}

\COeditorial
\COfloatPicture{uvodnikova.jpg}{}{r}{0.3\textwidth}{15}
{Milí čtenáři Corpus Omne,\\[4mm]
už je po Valentýnovi a tím pádem opět začal semestr. Abyste 
si mezi přednáškami, cvičeními a praktiky mohli chvíli 
odpočinout a přečíst si něco zábavného a obohacujícího, 
je tu další vydání Corpus Omne.

Tentokrát Vám přinášíme obsáhlý rozhovor s novým děkanem 
FJFI prof. Ing. Igorem Jexem, DrSc., jako obvykle si můžete 
přečíst o Bažantrikulaci a Drakiádě nebo se kouknout, 
jak se v podzimní sezoně (ne)dařilo našim kolegům sportovcům. 
Máme pro Vás dokonce i novinku v podobě jaderného komiksu.

Za celou redakci Vám přeji příjemné čtení a hodně štěstí 
do nového semestru.

\hfill Vaše šéfredaktorka

\hfill Hanča
\\[4mm]
PS: Kdybyste náhodou měli pocit, že máte nějak moc času 
a chuť něco hodnotného vytvářet, rádi Vás uvítáme v redakci. 
To víte, stárneme a potřebujeme nějakou novou krev... 
Bližší info na co@fjfi.cvut.cz.
}
\COLOAcategory{Šéfredaktorka}
\COsilentAuthor{Hana Wurzelová}
\COLOAcategory{Redaktoři}

\COtableofcontents
\pagebreak

\COsection{Sportem ku zdraví}

\COarticle{Jaderná Drakiáda}
\begin{COcolumn}
\lipsum
\COauthor{Lorem Ipsum}{lip}
\end{COcolumn}

\COarticle{Tralalalala}
\begin{COcolumn}[2]
\lipsum[4]
\COpicture{uvodnikova.jpg}{titulek}
\lipsum[5]
\COquote{Kdybyste náhodou měli pocit, že máte nějak moc času a chuť %
něco hodnotného vytvářet, rádi Vás uvítáme v redakci.}
\lipsum
\COauthor{Lorem Ipsum}{lip}
\end{COcolumn}

\COsection{Jádro \st{pudla} SUnie}

\COarticle{Víkendovka}
\begin{COcolumn}
\lipsum[1]
\COpicture{uvodnikova.jpg}{titulek}
\lipsum[2]
\end{COcolumn}
\COauthor{Lorem Ipsum}{lip}

\COarticle{Bažantrikulace}
\begin{COcolumn}
\lipsum[2]
\COquote{Kdybyste náhodou měli pocit, že máte nějak moc času a chuť %
něco hodnotného vytvářet, rádi Vás uvítáme v redakci.}
\lipsum[2]
\COauthor{Lorem Ipsum}{lip}
\end{COcolumn}
\vfill

\COLOAcategory{Korektura}
\COsilentAuthor{Lorem Ipsum}
\COLOAcategory{Zpracování}
\COsilentAuthor{Lorem Ipsum}
\COsilentAuthor{Lorem Ipsum (ilustrace na obálce)}

\pagebreak
\COsilentArticle{Pro chvíle oddechu a nudy}
\COfloatPicture{uvodnikova.jpg}{Sudoku}{r}{0.4\textwidth}{17}
\COmasthead

Oslavné básně nám pište na corpusomne@fjfi.cvut.cz
\end{document}
\end{verbatim}
%
\end{document}
