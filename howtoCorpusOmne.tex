\documentclass[a4paper]{article}

\usepackage{lipsum}
\usepackage[utf8]{inputenc}
\usepackage[IL2]{fontenc}
\usepackage[czech]{babel}

\usepackage{fullpage}

\title{Jak sázet \textbf{Corpus Omne} s balíčkem \texttt{CorpusOmne}}
\author{Eduard Šubert}

\begin{document}

\maketitle

\begin{abstract}
Dokumentace k balíčku \texttt{CorpusOmne}, který vznikl v roce 2014 na FJFI při ČVUT v Praze pro sázení fakultního časopisu Corpus Omne. Postupně jsou vysvětleny všechny příkazy a prostředí přidané balíčkem, včetně velkého příkladu na konci. Případné dotazy a připomínky směrujte na \texttt{suberedu@fjfi.cvut.cz}.
\end{abstract}

\section{Titulní strana}%%%%%%%%%%%%%%%%%%%%%%%%%%%%%%%%%%%
\begin{verbatim}
\begin{COtitlepage}[offset]{background}
\COtitle{Date}
\vfill
[anything]
\vfill
\COtitleBottom{citation}{author}
\end{COtitlepage}
\end{verbatim}

Prostředí \texttt{COtitlepage} vysází titulní stranu. Parametr \texttt{background} je obrázek na pozadí (poměr stran je zachován), volitelný parametr \texttt{offset} posunuje obrázek vertikálně.

Příkaz \texttt{COtitle} vysází hlavní titulek. Parametr \texttt{Date} slouží k označení vydání. Měl by být používán výhradně v prostředí \texttt{COtitlepage}.

Příkaz \texttt{COtitleBottom} vysází spodní citaci. Měl by být používán výhradně v prostředí \texttt{COtitlepage}. Parametr \texttt{citation} obsahuje citaci a parametr \texttt{author} obsahuje celé jméno autora citace.
Mezi \texttt{COtitle} a \texttt{COtitleBottom} může být libovolný obsah. 
\paragraph{Příklad:}
\begin{verbatim}
\begin{COtitlepage}[70]{titulka.jpg}
\COtitle{(2) 2013/2014}
\vfill
\COtitleBottom{... a tuto rovnici vyřešíme metodou uhodnutí.}{prof. Ing. Jiří Tolar, DrSc.}
\end{COtitlepage}
\end{verbatim}
%
\section{Editorial}%%%%%%%%%%%%%%%%%%%%%%%%%%%%%%%%%%%
\begin{verbatim}
\COeditorial
\end{verbatim}
Příkaz \texttt{COeditorial} vysází nadpis \textbf{editorial}, za ním následuje text (více v sekci \ref{article}).
%
\section{Obsah}%%%%%%%%%%%%%%%%%%%%%%%%%%%%%%%%%%%
Obsah je generován automaticky, každý článek je na úrovni \textbf{subsection}. Je tedy Nutné přidat dělení \textbf{section}:\\[0.3cm]
Příkaz \texttt{COsection} NEvysází nic. (přidá položku do Obsahu.)
\begin{verbatim}
\COsection{title}
\end{verbatim}
Příkaz \texttt{COtableofcontents} vysází Obsah.
\begin{verbatim}
\COtableofcontents
\end{verbatim}
%
\section{Tiráž}%%%%%%%%%%%%%%%%%%%%%%%%%%%%%%%%%%%
Tiráž se chová podobně jako Obsah, tedy obsahuje úrovně \textbf{section} (kategorie) a \textbf{subsection} (autoři).
Příkaz \texttt{COLOAcategory} NEvysází nic. (přidá novou kategorii do Tiráže.)
\begin{verbatim}
\COLOAcategory{category}
\end{verbatim}
Příkaz \texttt{COauthor} Vysází zkratku. Přidá do Tiráže jméno autora. \texttt{COauthor*} nepřidá záznam do tiráže (vhodné pro autora vícera článků).
\begin{verbatim}
\COauthor{name}{abb}
\end{verbatim}
Příkaz \texttt{COeditor} NEvysází nic. Přidá do Tiráže jméno editora bez zkratky.
\begin{verbatim}
\COeditor{name}
\end{verbatim}
Příkaz \texttt{COmasthead} vysází Tiráž.
\begin{verbatim}
\COmasthead
\end{verbatim}

\paragraph{Příklad:}
\begin{verbatim}
\COLOAcathegory{Redaktoři}
\COauthor{Jára Cimrman}{cim}
\COauthor*{Jára Cimrman}{cim}
\COeditor{Jára Cimrman}
\end{verbatim}
\paragraph{Doporučení:}
Je více než vhodné hned za obsah vložit inicializaci tiráže, například takto:
\begin{verbatim}
\COLOAcathegory{Šéfredaktorka}
\COeditor{Marie Curie}
\COLOAcathegory{Redaktoři}
\end{verbatim}
Nyní je možné na konci každého článku přidat autora(y) a tiráž bude správně formátována.
%
\section{Článek}%%%%%%%%%%%%%%%%%%%%%%%%%%%%%%%%%%%
\label{article}
Pro vysázení celého článku musíme použít několik příkazů:\\[0.3cm]
Příkaz \texttt{COarticle} vysází formátovaný nadpis \texttt{title} článku. Přidá záznam do obsahu. Volitelný parametr \texttt{abb} může sloužit jako kratší záznam do obsahu. \texttt{COarticle*} nepřidá záznam do obsahu.
\begin{verbatim}
\COarticle[abb]{title}
\end{verbatim}
Prostředí \texttt{COcolumn} slouží pro sázení textu do sloupců. Nepovinný parametr \texttt{number} nastavuje počet sloupců. Výchozí hodnota je $3$.
\begin{verbatim}
\begin{COcolumn}[number]
\end{COcolumn}
\end{verbatim}
Příkaz \texttt{COquote} vysází zvýrazněnou citaci \texttt{quote}. Měl by být používán výhradně v prostředí \texttt{COcolumn}.
\begin{verbatim}
\COquote{quote}
\end{verbatim}
Příkaz \texttt{COpicture} vysází obrázek \texttt{picture} volitelně s titulkem \texttt{title}.
\begin{verbatim}
\COpicture[title]{picture}
\end{verbatim}
\paragraph{Příklad}
\begin{verbatim}
\COarticle[Tralalala]{Tralalalalala}
\begin{COcolumn}
...
\COpicture[Jára Cimrman]{img/cimrman}
...
\COquote{Jára Cimrman ležící spící}
...
\COauthor{Marie Curie}{cur}
\COauthor*{Albert Einstein}{alb}
\end{COcolumn}
\end{verbatim}
%
\section{Ostatní}%%%%%%%%%%%%%%%%%%%%%%%%%%%%%%%%%%%
\subsection{Obrázky}
\begin{verbatim}
\COpagePicture{picture}
\end{verbatim}
Příkaz \texttt{COpagePicture} vysází obrázek \texttt{picture} na celou stranu. Obrázek je roztažen až do krajů, musí mít poměr stran 210:297.
\begin{verbatim}
\COfloatPicture{picture}{title}{align}{width}{lines}
\end{verbatim}
Příkaz \texttt{COfloatPicture} vysází obtékaný obrázek \texttt{picture} s titulkem \texttt{title}, zarovnáním \texttt{align} $ \in \left[r,l\right] $ a šířkou \texttt{width} (doporučuje se požívat tvar 0.4\\textwidth). Parametr \texttt{lines} je výška obrázku v počtu řádků (více parametr \texttt{lineheight} prostředí \texttt{wrapfigure}).
%
\subsection{Jaderka na síti}
\begin{verbatim}
\COsocial
\end{verbatim}
Příkaz \texttt{COsocial} vysází tabulku s odkazy na různé stránky spojené s Jaderkou a na sociální sítě.
\begin{verbatim}
\COzodiac
\end{verbatim}
Příkaz \texttt{COzodiac} vysází nadpis \textbf{Horoskop} i s ilustrací.


\end{document}
